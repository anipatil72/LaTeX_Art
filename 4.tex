\documentclass[11pt]{book}

\usepackage{sectsty}
\usepackage{graphicx}

\usepackage{blindtext} %This package generates automatic text
\usepackage{epigraph}

\title{Epigraph example}
\author{Overleaf}
\date{August 2021}


% Margins
\topmargin=-0.45in
\evensidemargin=0in
\oddsidemargin=0in
\textwidth=6.5in
\textheight=9.0in
\headsep=0.25in

\title{ \textbf{Some interesting Quotes for Probability!}}
\author{ Aniruddha Patil }
\date{\today}

\begin{document}
\frontmatter
\mainmatter
\maketitle
% \begin{document}
% \pagebreak

% Optional TOC
% \tableofcontents
% \pagebreak

%--Paper--

% \section{Section 1}

% Lorem Impsum


% \pagebreak
% \section{Section 2}
% Lorem Ipsum \\

%--/Paper--



\chapter{Axioms of Probability}
\epigraph{The most important problems of life are, for the most part, really only the problems of Probability.}{\textit{Laplace}}

% \part*{Basics of Probability}
\section{Basics of Probability}
 {This is all very standard nothing out of the ordinary. The theory of Probability arised from the game of chance. In order to predict the winners.}

$$A_1 + A_2 + A_3 + A_4 + \ldots + A_N = \mathbf{XOR(A_1+A_2+A_3+A_4+\ldots+A_N)}$$

{{\hspace*{-0.5cm}}The classical definition of Probability is given as : }
$$P(A) = \frac{Number of events favourable to outcome A}{Total Number of Possible events}$$

{Thanks to \LaTeX}



% \blindtext
\end{document}